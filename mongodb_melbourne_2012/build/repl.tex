\documentclass[letter]{article}
% generated by Docutils <http://docutils.sourceforge.net/>
\usepackage{fixltx2e} % LaTeX patches, \textsubscript
\usepackage{cmap} % fix search and cut-and-paste in Acrobat
\usepackage{ifthen}
\usepackage[T1]{fontenc}
\usepackage[utf8]{inputenc}
\setcounter{secnumdepth}{0}

%%% Custom LaTeX preamble
\usepackage{fullpage} \usepackage{parskip} \usepackage{fancyhdr} \usepackage{graphicx} \pagestyle{fancy} \renewcommand{\headrulewidth}{0pt} \renewcommand{\footrulewidth}{0pt}

%%% User specified packages and stylesheets

%%% Fallback definitions for Docutils-specific commands

% hyperlinks:
\ifthenelse{\isundefined{\hypersetup}}{
  \usepackage[colorlinks=true,linkcolor=blue,urlcolor=blue]{hyperref}
  \urlstyle{same} % normal text font (alternatives: tt, rm, sf)
}{}
\hypersetup{
  pdftitle={Replication Notes},
}

%%% Title Data
\title{\phantomsection%
  Replication Notes%
  \label{replication-notes}}
\author{}
\date{}

%%% Body
\begin{document}
\maketitle


\section{Starting a Master/Slave Pair%
  \label{starting-a-master-slave-pair}%
}

Create directories for each mongod instance:
%
\begin{quote}{\ttfamily \raggedright \noindent
mkdir~/data/master~/data/slave
}
\end{quote}

Start the master and the slave:
%
\begin{quote}{\ttfamily \raggedright \noindent
mongod~-{}-dbpath~/data/master~-{}-port~27017~-{}-master\\
~\\
\#~You~may~have~to~replace~"localhost"~with~your~machine's~hostname.\\
mongod~-{}-dbpath~/data/slave~-{}-port~27018~-{}-slave~-{}-source~localhost:27017
}
\end{quote}


\section{Starting a Replica Set%
  \label{starting-a-replica-set}%
}

Create directories for each mongod instance:
%
\begin{quote}{\ttfamily \raggedright \noindent
mkdir~/data/rs0~/data/rs1~/data/rs2
}
\end{quote}

Start three instances of mongod:
%
\begin{quote}{\ttfamily \raggedright \noindent
mongod~-{}-dbpath~/data/rs0~-{}-port~27017~-{}-replSet~myset\\
~\\
mongod~-{}-dbpath~/data/rs1~-{}-port~27018~-{}-replSet~myset\\
~\\
mongod~-{}-dbpath~/data/rs2~-{}-port~27019~-{}-replSet~myset
}
\end{quote}

Connect to the any mongod instance:
%
\begin{quote}{\ttfamily \raggedright \noindent
mongo~-{}-port~27018
}
\end{quote}

Initialize the replica set:
%
\begin{quote}{\ttfamily \raggedright \noindent
>~rs.initiate()
}
\end{quote}

After a few seconds the mongod process you are connected to will become
the replica set primary. The prompt will change to ``PRIMARY>''. Run rs.status()
to see the current configuration:
%
\begin{quote}{\ttfamily \raggedright \noindent
PRIMARY>~rs.status()\\
\{\\
~~~~~~"set"~:~"myset",\\
~~~~~~"date"~:~ISODate("2012-05-11T17:56:37Z"),\\
~~~~~~"myState"~:~1,\\
~~~~~~"members"~:~{[}\\
~~~~~~~~~~~~~~\{\\
~~~~~~~~~~~~~~~~~~~~~~"\_id"~:~0,\\
~~~~~~~~~~~~~~~~~~~~~~"name"~:~<your~hostname>:27018",\\
~~~~~~~~~~~~~~~~~~~~~~"health"~:~1,\\
~~~~~~~~~~~~~~~~~~~~~~"state"~:~1,\\
~~~~~~~~~~~~~~~~~~~~~~"stateStr"~:~"PRIMARY",\\
~~~~~~~~~~~~~~~~~~~~~~"optime"~:~\{\\
~~~~~~~~~~~~~~~~~~~~~~~~~~~~~~"t"~:~1336758831000,\\
~~~~~~~~~~~~~~~~~~~~~~~~~~~~~~"i"~:~1\\
~~~~~~~~~~~~~~~~~~~~~~\},\\
~~~~~~~~~~~~~~~~~~~~~~"optimeDate"~:~ISODate("2012-05-11T17:53:51Z"),\\
~~~~~~~~~~~~~~~~~~~~~~"self"~:~true\\
~~~~~~~~~~~~~~\}\\
~~~~~~{]},\\
~~~~~~"ok"~:~1\\
\}
}
\end{quote}

Add the other members:
%
\begin{quote}{\ttfamily \raggedright \noindent
PRIMARY>~rs.add('<your~hostname>:27017')\\
PRIMARY>~rs.add('<your~hostname>:27019')
}
\end{quote}


\section{Config Objects%
  \label{config-objects}%
}

There are many optional settings that can be configured using the config
object:
%
\begin{quote}{\ttfamily \raggedright \noindent
\{\\
~~\_id~:~<setname>,\\
~~members:~{[}\\
~~~~\{\\
~~~~~~\_id~:~<ordinal>,\\
~~~~~~host~:~<hostname{[}:port{]}>\\
~~~~~~{[},~arbiterOnly~:~true{]}\\
~~~~~~{[},~buildIndexes~:~<bool>{]}\\
~~~~~~{[},~hidden~:~true{]}\\
~~~~~~{[},~priority:~<priority>{]}\\
~~~~~~{[},~tags:~\{loc1~:~desc1,~loc2~:~desc2,~...,~locN~:~descN\}{]}\\
~~~~~~{[},~slaveDelay~:~<n>{]}\\
~~~~~~{[},~votes~:~<n>{]}\\
~~~~\}\\
~~~~,~...\\
~~{]},\\
~~{[}settings:~\{\\
~~~~{[}getLastErrorDefaults:~<lasterrdefaults>{]}\\
~~~~{[},~getLastErrorModes~:~<modes>{]}\\
~~{]}\\
\}
}
\end{quote}

A quick example:
%
\begin{quote}{\ttfamily \raggedright \noindent
>~conf~=~\{~\_id:~'myset',\\
...~~~~~~~~members:~{[}\\
...~~~~~~~~~~\{~\_id:~0,~host:~'<your~hostname>:27017'\},\\
...~~~~~~~~~~\{~\_id:~1,~host:~'<your~hostname>:27018'\},\\
...~~~~~~~~~~\{~\_id:~2,~host:~'<your~hostname>:27019'\}\\
...~~~~~~~~{]}\\
...~~~~~~\}\\
~\\
>~rs.initiate(conf)
}
\end{quote}

To reconfigure the set:
%
\begin{quote}{\ttfamily \raggedright \noindent
PRIMARY>~conf~=~rs.conf()\\
PRIMARY>~conf.members{[}2{]}.priority~=~100\\
PRIMARY>~rs.reconfig(conf)
}
\end{quote}

To remove an option set it to its default setting:
%
\begin{quote}{\ttfamily \raggedright \noindent
PRIMARY>~conf~=~rs.conf()\\
PRIMARY>~conf.members{[}2{]}.priority~=~1\\
PRIMARY>~rs.reconfig(conf)
}
\end{quote}


\section{Other Important Commands%
  \label{other-important-commands}%
}
%
\begin{quote}{\ttfamily \raggedright \noindent
rs.help()\\
rs.status()\\
rs.slaveOk()\\
db.printReplicationInfo()\\
db.printSlaveReplicationInfo()
}
\end{quote}


\section{Exercises%
  \label{exercises}%
}
\newcounter{listcnt0}
\begin{list}{\arabic{listcnt0}.}
{
\usecounter{listcnt0}
\setlength{\rightmargin}{\leftmargin}
}

\item Set up a replica set using the steps above.

\item Run the command to step down the primary: db.runCommand(\{ replSetStepDown: 1\}); Ensure that a secondary node is elected as the new primary.

\item Practice automated failover. In this case, you'll want to terminate the primary node manually.

\item Add a node to an existing live replica set. This involves setting up a new node and either running rs.add() from the shell or, on a lower level, running the replicaSetReconfig command.
\end{list}

\end{document}
