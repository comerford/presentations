\documentclass[letter]{article}
% generated by Docutils <http://docutils.sourceforge.net/>
\usepackage{fixltx2e} % LaTeX patches, \textsubscript
\usepackage{cmap} % fix search and cut-and-paste in Acrobat
\usepackage{ifthen}
\usepackage[T1]{fontenc}
\usepackage[utf8]{inputenc}
\setcounter{secnumdepth}{0}

%%% Custom LaTeX preamble
\usepackage{fullpage} \usepackage{parskip} \usepackage{fancyhdr} \usepackage{graphicx} \pagestyle{fancy} \renewcommand{\headrulewidth}{0pt} \renewcommand{\footrulewidth}{0pt}

%%% User specified packages and stylesheets

%%% Fallback definitions for Docutils-specific commands

% hyperlinks:
\ifthenelse{\isundefined{\hypersetup}}{
  \usepackage[colorlinks=true,linkcolor=blue,urlcolor=blue]{hyperref}
  \urlstyle{same} % normal text font (alternatives: tt, rm, sf)
}{}
\hypersetup{
  pdftitle={Sharding Notes},
}

%%% Title Data
\title{\phantomsection%
  Sharding Notes%
  \label{sharding-notes}}
\author{}
\date{}

%%% Body
\begin{document}
\maketitle


\section{Setting up a sharded cluster%
  \label{setting-up-a-sharded-cluster}%
}

Create directories for mongod instances:
%
\begin{quote}{\ttfamily \raggedright \noindent
mkdir~/data/shard1~/data/shard2~/data/config
}
\end{quote}

Start up 3 mongod instances:
%
\begin{quote}{\ttfamily \raggedright \noindent
mongod~-{}-dbpath~/data/shard1~-{}-port~10000~-{}-shardsvr\\
mongod~-{}-dbpath~/data/shard2~-{}-port~10001~-{}-shardsvr\\
mongod~-{}-dbpath~/data/config~-{}-port~20000~-{}-configsvr
}
\end{quote}

Start an instance of mongos:
%
\begin{quote}{\ttfamily \raggedright \noindent
\#~-{}-chunkSize~1~means~use~a~1MB~chunk~size.~This~is~for\\
\#~demonstration~purposes.\\
mongos~-{}-configdb~localhost:20000~-{}-chunkSize~1
}
\end{quote}

Add shards to the cluster:
%
\begin{quote}{\ttfamily \raggedright \noindent
mongo\\
mongos>~use~admin\\
~\\
\#~Pre~MongoDB~2.0\\
mongos>~db.runCommand(\{'addshard':~'localhost:10000'\})\\
mongos>~db.runCommand(\{'addshard':~'localhost:10001'\})\\
~\\
\#~Post~MongoDB~2.0\\
mongos>~sh.addShard('localhost:10000')\\
mongos>~sh.addShard('localhost:10001')
}
\end{quote}

Enable sharding on a database:
%
\begin{quote}{\ttfamily \raggedright \noindent
\#~Pre~MongoDB~2.0\\
mongos>~db.runCommand(\{'enablesharding':~<database~name>\})\\
~\\
\#~Post~MongoDB~2.0\\
mongos>~sh.enableSharding(<database~name>)
}
\end{quote}

Shard a collection:
%
\begin{quote}{\ttfamily \raggedright \noindent
\#~Pre~MongoDB~2.0\\
mongos>~db.runCommand(\{'shardcollection':~<namespace>,~'key':~<shard~key>\})\\
~\\
\#~Post~MongoDB~2.0\\
mongos>~sh.shardCollection(<namespace>,~<shard~key>)
}
\end{quote}


\section{Important Sharding Commands%
  \label{important-sharding-commands}%
}
%
\begin{quote}{\ttfamily \raggedright \noindent
sh.help()\\
db.printShardingStatus()
}
\end{quote}


\section{Exercises%
  \label{exercises}%
}
\newcounter{listcnt0}
\begin{list}{\arabic{listcnt0}.}
{
\usecounter{listcnt0}
\setlength{\rightmargin}{\leftmargin}
}

\item Set up a sharded cluster on your local machine using the instructions above.

\item Generate some simple data using the example from the ``General Ops Notes'' page.

\item Shard the collection on a field (or fields) other than \_id.

\item Explore the config db. Query the changelog and locks collections.
\end{list}

\end{document}
